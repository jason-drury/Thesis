%%%%%%%%%%%%%%%%%%%%%%%%%%%%%%%%%%%%%%%%%%%%%%%%%%%%%%%%%%%%%%%%%%%%%%%%%%%
%% This is Jamie's PhD thesis style declarations                                                                                             %%
%% If you're looking for a template you've come to the right place =)                                                                        %%
%% I've commented each section so only take what you need and understand what you've taken!                                                  %%
%% Also don't take the header/footer declarations without asking me cause it took ages to figure out and I want my thesis to look unique ;)  %%
%%%%%%%%%%%%%%%%%%%%%%%%%%%%%%%%%%%%%%%%%%%%%%%%%%%%%%%%%%%%%%%%%%%%%%%%%%%

\usepackage[times]{quotchap}  	%% chapter headings
\usepackage{color}                 						%% color latex stuff
\usepackage{amsmath}               					%% lots of maths symbols =)
\usepackage{amssymb}              					%% even more maths symbols =)
\usepackage{subfigure}             					%% lets you put more than one graphic into a figure, but not terribly useful
\usepackage{xspace}                						%% this package detects whether to put space in or not - I don't think it works very well
\usepackage{listings}              						%% for putting some program code into the document
\usepackage{graphicx}
\usepackage{epsfig}                						%% lets one put (e)ps graphics into the document
\usepackage{hhline}                						%% need these to get nice lines in some tables
\usepackage{array}                 						%% makes some cool alterations to the tabular environment
\usepackage{booktabs}              					%% for even prettier tables... yes it is possible!
\usepackage{relsize}               						%% allow relative text scaling
\usepackage{fancybox}              					%% love them fancy boxes
\usepackage{equationformat}        			%% creates the little ovaloidish (!) boxes around my equation numbers (requires amsmath and fancybox)
\usepackage{multirow}              					%% make tables where columns can go over many rows
\usepackage[nottoc,notlot,notlof]{tocbibind}             %% now the bibliography shows up in the table of contents
\usepackage{times}
\usepackage{float} 
\usepackage{aastex_hack}
\usepackage{setspace}
\usepackage{todonotes}
\usepackage{appendix}
\usepackage{setspace}
\usepackage{layout} 
\usepackage{wrapfig}
\usepackage{tabularx}
\usepackage{bm}
\usepackage{lscape}
\usepackage{rotating}
\usepackage{multirow}
\usepackage{footnote}
\usepackage{longtable}
\usepackage{geometry}
\usepackage{pdflscape}

%%%%%%%%%%%%%%%%%%%%%%%%%%%%%%%%%%%%%%%%%%%%%%%%%%%%%%%%%%%%%%%%%%%%%%%%%%%%%%%%

\usepackage{setspace}              %% how about that spacing...
\onehalfspacing                    %% lots of space between lines to make it nice and readable (needs setspace) - requirement of PhD theses

%%%%%%%%%%%%%%%%%%%%%%%%%%%%%%%%%%%%%%%%%%%%%%%%%%%%%%%%%%%%%%%%%%%%%%%%%%%%%%%%

\setlength{\oddsidemargin}{0.7cm}  %% change the margins on the pages to give 30mm of white space on each side - requirement of PhD theses
\setlength{\evensidemargin}{0.3cm}
\setlength{\textwidth}{14.9cm}
\setlength{\textheight}{22cm}

%%%%%%%%%%%%%%%%%%%%%%%%%%%%%%%%%%%%%%%%%%%%%%%%%%%%%%%%%%%%%%%%%%%%%%%%%%%%%%%%

\renewcommand{\floatpagefraction}{0.6}  %% make it so that 60% of the page must be taken up by a float before it becomes a float page

%%%%%%%%%%%%%%%%%%%%%%%%%%%%%%%%%%%%%%%%%%%%%%%%%%%%%%%%%%%%%%%%%%%%%%%%%%%%%%%%

\usepackage{natbib}              %% the bibliography stylin'
\bibpunct[, ]{(}{)}{;}{a}{}{,}     %% oh yeah, I could tell you what this does, but nah! (needs natbib)

%%%%%%%%%%%%%%%%%%%%%%%%%%%%%%%%%%%%%%%%%%%%%%%%%%%%%%%%%%%%%%%%%%%%%%%%%%%%%%%

\usepackage[calcwidth]{titlesec}   %% This next bit changes the section titles to be pretty
\titleformat{\section}[hang]{\scshape\bfseries}{\Large\thesection}{12pt}{\Large}[{\titlerule[0.5pt]}]
\titleformat{\subsection}[hang]{\scshape\bfseries}{\thesubsection}{.5em}{}

%%%%%%%%%%%%%%%%%%%%%%%%%%%%%%%%%%%%%%%%%%%%%%%%%%%%%%%%%%%%%%%%%%%%%%%%%%%%%%%%

\usepackage{calc}                  													%% We need to calculate a width to push the right boxes over by.
\newlength{\rightboxlength}
\setlength{\rightboxlength}{\textwidth-14mm} 		%% the box is 10mm wide, the rules are 4mm, so 14mm in total
\definecolor{pageboxcolor}{rgb}{9,9,9}  						%% this is the colour of the boxes at the bottom
\usepackage{fancyhdr}              											%% I like them nice looking headers and footers - this section defines the page prettiness =)
\fancyhead{}                       														%% first reset the headers and footers
\fancyhead[RO]{\em \rightmark}     								%% make the odd pages have the section name on the top right
\fancyhead[LE]{\em \leftmark}      									%% make the even pages have the chapter name on the top left

\fancyfoot{}
\fancyfoot[LE]{\fcolorbox{white}{white}{\parbox[c][5mm]{10mm}{\rule{0cm}{0mm}\color{black}{\begin{center}\bfseries {\rm\thepage}\end{center}}}}}    %% page nums on the bottom in a nice box - this is hard...
\fancyfoot[RO]{\hspace*{\rightboxlength}\fcolorbox{white}{white}{\parbox[c][5mm]{10mm}{\rule{0cm}{0mm}\color{black}{\begin{center}\bfseries{\rm\thepage}\end{center}}}}}


\renewcommand{\footrulewidth}{0.4pt}
\renewcommand{\footruleskip}{0mm}
\pagestyle{fancy}                  			%% bring all the stylin' into effect (must come after all the fancyhead and fancyfoot stuff)
\fancypagestyle{plain}{%           %% now redefine the plain style pages (chapter pages, contents pages) to have the same page number stuff on the bottom
	\fancyhf{}
	\fancyfoot[RO]{\hspace*{\rightboxlength}\fcolorbox{white}{pageboxcolor}{\parbox[c][5mm]{10mm}{\rule{0cm}{0mm}\color{black}{\begin{center}\bfseries \thepage\end{center}}}}}
	\renewcommand{\headrulewidth}{0pt}
	\renewcommand{\footrulewidth}{0.5pt}
}
\makeatletter                      %% this next section (till \makeatother) makes sure that blank pages are actually completely blank, cause they're not usually
\def\cleardoublepage{\clearpage\if@twoside \ifodd\c@page\else
	\hbox{}
	\vspace*{\fill}
	\thispagestyle{empty}
	\newpage
	\if@twocolumn\hbox{}\newpage\fi\fi\fi}
\makeatother

\newlength{\numberwidth}           %% this next section is bad - a terrible hack for the compact group chapter - don't use it unless it's what you really want
\settowidth{\numberwidth}{{\tiny 0}}
\newcommand{\nw}{\hspace*{\numberwidth}}
\newlength{\onewidth}
\settowidth{\onewidth}{1}
\newcommand{\ow}{\hspace*{\onewidth}}
\newlength{\dotwidth}
\settowidth{\dotwidth}{{\tiny .}}
\newcommand{\dw}{\hspace*{\dotwidth}}
\newlength{\minuswidth}
\settowidth{\minuswidth}{$-$}
\newcommand{\mw}{\hspace*{\minuswidth}}
\newlength{\asteriskwidth}
\settowidth{\asteriskwidth}{*}
\newcommand{\aw}{\hspace*{\asteriskwidth}}
\newlength{\comparisonwidth}
\settowidth{\comparisonwidth}{$\leq$}
\newcommand{\cw}{\hspace*{\comparisonwidth}}%
\newlength{\pmwidth}
\settowidth{\pmwidth}{$\pm$}
\newcommand{\pw}{\hspace*{\pmwidth}}

%-------------------------------------------------------------------------------------------------------------------------------------------------------------------------------------------------------------%
%-------------------------------------------------------------------------------------------------------------------------------------------------------------------------------------------------------------%
% --------------  JFK-specific commands

\newcommand{\toDo}{\todo[inline]}
\def\HI{H{\textsc{i}}~} 
\def\radm{\,rad\,m$^{-2}$~}
\def\lamsq{$\lambda^2$~}
\def\los{line-of-sight}
\def\Ref{{\color{green}(\textbf{ref})}}
\def\chisq{$\chi^2_r$}
\def\etc{{\color{red}$\mathbf{\dots}$}}
\def\e363{PKS\,J0548$-$3257}
\def\mjyb{mJy\,b$^{-1}$\,}
\def\n612{NGC\,612}


%-------------------------------------------------------------------------------------------------------------------------------------------------------------------------------------------------------------%
%-------------------------------------------------------------------------------------------------------------------------------------------------------------------------------------------------------------% 

\def\kms{km~${\rm s}^{-1}$\xspace}  %% now a few definitions of my own to make life easier
\def\mHI{\rm H\mathsmaller{I}}      %% math mode version of HI


\newcommand{\expect}[1]{\langle{#1}\rangle}
\renewcommand{\labelenumi}{(\arabic{enumi})}

\renewcommand{\thefootnote}{\fnsymbol{footnote}}  %% make footnote symbols like dagger, and asterisk instead of numbers

\usepackage{titlepage_sydney_uni}  %% my redefinition of \maketitle to put the UniMelb logo on the title-page, along with a nice layout
                                      %% the option [acidfree] for this package puts the text "Produced on acid-free paper" at the bottom

\newenvironment{nopointitemize}{\renewcommand\labelitemi{}\begin{itemize}}{\end{itemize}}  %% make an itemize environment with no little point

