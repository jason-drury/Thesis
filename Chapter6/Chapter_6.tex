\chapter{Conclusions and Outlook}

In this thesis we investigated the stellar membership of the open clusters in the nominal \Kepler{} FoV. We used a combination of photometric, astrometric and asteroseismic data to improve these cluster membership determinations, and produced a complete database of cluster members with a focus on the ensemble analysis of these cluster members. We also presented the analysis of a selection of non-member variable stars within the fields of view for these clusters. We summarise the primary results and conclusions below:

\begin{itemize}
    \item We applied unsupervised clustering algorithms to the \Gaia{} DR2 astrometric data to produce complete cluster membership determinations for all stars within a radius of one degree for each of the four open clusters in the nominal \Kepler{} field of view. We selected a Gaussian Mixture Model as the best investigated clustering method for simultaneously accounting for differing cluster densities as well as accounting for the background field distribution. We produced colour-magnitude diagrams for each of these clusters showing the distribution of likely cluster members in relation to the surrounding field star distribution. These represent the most complete, reliable membership determinations for these clusters, with our work incorporating a rigorous exploration of the heteroskedastic uncertainty parameter space. We also cross-matched these \Gaia{} membership determinations with the \Kepler{} Input Catalog, including the non-targeted \Kepler{} stars.
    
    \item Using these membership determinations we produced target lists that Isabel Colman used for extracting cluster member light curves from the \Kepler{} superstamp images of NGC\,6791 and NGC\,6819. We investigated three main methods for correcting the systematic trends within these raw image-subtracted light curves, including high-pass filtering, polynomial de-trending and principle component analysis. We produced custom co-trending basis vectors to be used for correcting these light curves, with an emphasis on producing corrected light curves for the cluster red giant members. This has enabled us to conduct ensemble analyses of the full set of cluster red giants.

    \item We presented new red giant members with previously undetected oscillation signatures, and extracted the global asteroseismic parameters of \numax{}, \dnu{} and $\epsilon$ for these stars. We compiled the most complete sample of cluster red giants for NGC\,6791 and NGC\,6819 to date, and updated values for some of these previously reported variables based on the additional data available through the \Kepler{} superstamps. We presented the cluster ensemble for the first time in the context of the entire nominal \Kepler{} red giant sample and produced updated asteroseismic diagrams showing the relationship between \numax{} and \dnu{}, and $\epsilon$, with cluster isochrones clearly visible for each of these ensembles.

    \item We led an investigation into the variability of the previously unclassified rotationally-modulated star, KIC\,2569073, where we were responsible for extracting and processing the raw aperture photometry from the superstamp image of NGC\,6791. We conducted the search for rapid oscillations above the noise level in the power spectrum that resulted in a non-detection result and the classification of the star as an $\alpha^2 $\,CVn star.
\end{itemize}

This thesis has been possible due to the exceptional quality of data available from the \Kepler{} and \Gaia{} space telescopes. Whilst the \Kepler{} space telescope has been retired following the depletion of its fuel supply, the data collected during its lifetime continues to reveal insights into stellar populations and their interiors. In particular the superstamps covering NGC\,6791 and NGC\,6819 have only recently been processed, providing an opportunity to analyse additional non-targeted stars and longer time series for partially-targeted stars in these clusters. These newly extracted stars include a number of eclipsing binaries, blue straggler stars, and rotationally-modulated variables with over 4 years of observations that may help enhance our understanding of long-term apsidal motion, stellar evolution in binary pairs, and stellar activity and rotation rates. 

We have identified a number of rotational variables and provided approximate rotational periods, however the 4 year observational baseline provides a valuable opportunity to more rigorously model stellar surface activity, and the opportunity to obtain rates of differential latitudinal rotation, along with the relative strength and/or size of the stellar spots.

During our investigation of the global asteroseismic parameters in cluster red giant stars we utilised Gaussian Processes to directly fit the red giant excess in the time domain. We attempted to improve this process to fit a series of individual modes simultaneously, but were unable to complete this model. We hope to revisit this idea in the future as it has the potential to enable the extraction of asteroseismic parameters in low signal-to-noise regimes, with rigorous inference of uncertainties. This will be particularly useful in regards to \textsc{Tess} observations of red giants, where the photometric detection limit is much brighter than \Kepler{}.

Our cluster membership determinations based on \Gaia{} astrometric observations provided the ability to distinguish field stars from likely cluster members, however we later found a small number of these stars needed to be re-classified based on photometric and asteroseismic measurements. We believe this is primarily a result of the unconstrained parallax measurements for these stars, but also acknowledge the Gaussian Mixture Model is not a perfect description of open cluster morphology. Instead, future investigations could use more physical based mixture models, such as an ellipsoid to describe the cluster spatial distribution component \citep{kuhn_mixture_2017} and a large-scale galactic density model to better describe the field star distribution, instead of simple Gaussians.

These are a small fraction of the potential areas for future work, and with the upcoming \Gaia{} data release 3 and \textsc{Tess} providing improved (almost)-full-sky coverage with a combination of astrometric, photometric, and asteroseismic data the future is bright! 