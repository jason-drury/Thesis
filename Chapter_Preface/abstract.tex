\chapter*{\textsc{Abstract}}

\vspace{1.5cm}

Asteroseismology, the study of stellar interiors by measuring changes in the brightness of a star, has taken off over the last decade thanks to the unprecedented quantity and quality of photometric data from space telescopes such as \Kepler{} and \textsc{CoRoT}. Open clusters are particularly important, providing ideal targets for understanding stellar evolution through ensemble analyses. These stars are believed to have common ages and metallicities, constraining the parameters affecting their evolution mainly to their mass. 

This thesis presents an analysis of the open cluster members and the surrounding field stars for the four open clusters within the nominal \Kepler{} field of view. %In Chapter 2 we present the initial extraction attempts from the superstamp images of the cluster centres and analyse some of the highly variable stars in these fields of view.
Prior to conducting any ensemble analysis, we must first separate those stars that are cluster members from foreground and background contaminants. With the release of high precision proper motions of 1.6 billion stars from the \Gaia{} space telescope we can now determine this membership for large numbers of stars. We have used unsupervised machine learning algorithms including Gaussian mixture models on \Gaia{} DR2 data to determine cluster membership scores for stars in these four open clusters. We present the results of this membership analysis, including databases of likely members, and use these membership determinations to investigate the red giant solar-like oscillators in NGC\,6791 and NGC\,6819 as an ensemble. We present global asteroseismic properties of all cluster red giants that have measurable oscillations, and include asteroseismic diagrams placing the cluster isochrones in the perspective of the full \Kepler{} red giant sample for the first time. Our identification and classification of other variable stars within the fields of these clusters are also presented, including an in-depth analysis of a previously unknown $\alpha^2$CVn variable that is suitable for future investigations of Ap magnetic fields.