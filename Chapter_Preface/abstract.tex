\chapter*{\textsc{Abstract}}

\vspace{1.5cm}

Asteroseismology, the study of stellar interiors by measuring changes in the brightness of a star, has taken off over the last decade thanks to the unprecedented quantity and quality of photometric data from space telescopes such as Kepler and CoRoT. Of all the stars with such data, those in open clusters are arguably the most important - providing ideal targets for understanding stellar evolution through ensemble analyses. These stars are believed to have common ages and metallicities, constraining the parameters affecting their evolution to their mass. 

This thesis presents an analysis of the stars in the four open clusters in the nominal Kepler field of view. Prior to conducting any ensemble analysis, we must first separate those stars that are cluster members from foreground and background contaminants. With the release of high precision proper motions of 1.6 billion stars from the Gaia Space Telescope we can now constrain this membership. We have used unsupervised machine learning algorithms including gaussian mixture models on Gaia DR2 data to determine cluster membership probabilities for stars in these four open clusters. We present the results of this membership analysis, including databases of likely members. 

We investigate the angle of inclination of the red giant members of these four clusters and confirm (?) \cite{corsaro_spin_2017} the inherent bias in the angle of inclination originating from the 
